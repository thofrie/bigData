\chapter{Einleitung}
\label{sec:einleitung}

In der heutigen Zeit gibt es zahlreiche Online-Shops. Das Angebot ist groß und vielfältig. Neben subjektiven Eigenschaften des Anbieters wie Gestaltung bzw. Design der Website, sind es vor allem objektive Eigenschaften wie die Art und Größe des Produktsortiments, welche den Kunden zum Kauf bewegen. Eine Erstbestellung im Shop führt jedoch nicht zwangsweise zu einer Folgebestellung.\\

Umso wichtiger ist es daher, eine Kundenbindung aufzubauen. Hierdurch ist es möglich, den Kunden trotz vielfältiger Konkurrenzangebote zum Wiederkauf zu bewegen.  Es gibt verschiedene Wege, um dies zu erreichen. Eine verbreitete Methode ist die Anreizsetzung mit monetären Vorteilen wie speziellen Aktionen, Rabatten oder Gutscheinen.\\

Dabei ergibt sich eine Problemstellung. Erhalten alle bisherigen Kunden monetäre Vorteile, so betrifft dies auch Kunden, welche ohnehin bereit gewesen wären, erneut über die Plattform einzukaufen. Damit handelt es sich um direkte Gewinneinbußen. Eine Abwägung, bei welchen Kunden eine Anreizsetzung sinnvoll ist und bei welchen Kunden nicht, ist somit entscheidend für den Erfolg einer Marketingkampagne.\\

Im vorliegenden Fall hat sich ein Online-Händler von analogen wie auch digitalen Medien sowie dazu passendem Zubehör dazu entschieden mittels Gutscheinen Kunden zu einem Wiederkauf zu bewegen. Hier muss entschieden werden, ob ein Kunde einen Gutschein erhalten soll oder nicht. Ziel ist es, den Umsatz des Online-Händlers zu maximieren.\\

Die Umsatzmaximierung soll mit Methoden des Maschinellen Lernens und des Data Minings erreicht werden. Dabei geht es um das Trainieren von Algorithmen zur binären Klassifikation, welche eine intelligente Vergabe von Gutscheinen erreichen sollen. Diese sollen automatisiert Regeln ableiten, um Kunden anhand gegebener Merkmale in Wiederkäufer und Nicht-Wiederkäufer einzuordnen.\\ 

Als Trainingsdatenbasis steht ein Teildatensatz des Data Mining Cups 2010 zur Verfügung. Die Validierung und die Umsatzberechnung erfolgt, mithilfe einer vorgegeben Kostenmatrix, auf dem extern bereitgestellten Testdatensatz. Die Maximierung des Umsatzes soll anhand eines strukturierten Vorgehens nach Vorbild des CRISP-DM Modells erreicht werden.\\ 

